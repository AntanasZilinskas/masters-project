%--------------------------------------------------------------------
\subsection{From \textit{SolarKnowledge} to \textsc{EVEREST}}
\label{sec:sk2ev-transition}
%--------------------------------------------------------------------
Despite achieving high overall accuracy (99.96\%) on M5-72h tasks, evaluation of the \textit{SolarKnowledge} prototype revealed critical performance limitations that undermined its operational viability.

\textbf{(i) Poor precision and excessive false alarms.}
The \textit{SolarKnowledge} model exhibits problematic precision of only \textbf{82.91\%} on the M5-72h benchmark, meaning that nearly one in five positive predictions is a false alarm. For operational space weather forecasting, where false alerts trigger costly mitigation measures and erode forecaster confidence, this false alarm rate is operationally unacceptable. The high accuracy (99.96\%) masks this precision deficit due to the extreme class imbalance in solar flare data.

\textbf{(ii) Lack of confidence assessment.}
With no uncertainty quantification or calibration measurement, \textit{SolarKnowledge} provides no mechanism to distinguish high-confidence predictions from uncertain ones. Operational forecasters require confidence estimates to make informed decisions about alert thresholds and mitigation strategies, yet the baseline architecture offers no reliability assessment.

\textbf{(iii) Parameter inefficiency.}
At 1.999M parameters, the \textit{SolarKnowledge} architecture requires substantial computational resources while delivering suboptimal precision performance. For deployment in operational forecasting systems with latency and resource constraints, a more efficient architecture was essential.

These performance deficits motivated the transition to \textsc{EVEREST}, designed to address precision limitations through architectural innovations while adding uncertainty quantification capabilities.

\paragraph{Temporal focusing through a learnable bottleneck.}
In \textsc{EVEREST}, global-average pooling is replaced by a single-query attention kernel that assigns an importance weight $\alpha_t\!\in\![0,1]$ to every ten-minute snapshot in the SHARP sequence. The resulting summary $z=\sum_{t}\alpha_{t}h_{t}$ enables focused temporal modeling that reduces spurious activations contributing to false alarms.

\paragraph{Precision-aware evidential learning.}
To address both precision and uncertainty deficits, a Normal–Inverse–Gamma head predicts four natural parameters $\{\mu,v,\alpha,\beta\}$, from which a conjugate Beta distribution over forecast probabilities is recovered in closed form \citep{sensoy2018evidential}. The evidential framework naturally handles uncertainty, allowing the model to abstain from low-confidence predictions that would otherwise contribute to false alarms. This provides principled uncertainty quantification with an Expected Calibration Error (ECE) of \textbf{0.036} on the M5–72h task.

\paragraph{Extreme-risk awareness through an EVT tail.}
The Extreme Value Theory (EVT) head fits a Generalised Pareto Distribution (GPD) to logit exceedances above the batchwise 90$^{\mathrm{th}}$ percentile \citep{coles2001extremes}, providing explicit modeling of rare, high-magnitude events. By focusing on extreme values rather than all positive cases, this component helps reduce false alarms from moderate-intensity events while maintaining sensitivity to genuine high-risk scenarios.

\paragraph{Precision-focused composite loss.}
The focal, evidential and EVT objectives are combined with a precursor loss term:
\[
  \mathcal{L}=0.8\,\mathcal{L}_{\text{focal}}
             +0.1\,\mathcal{L}_{\text{evid}}
             +0.1\,\mathcal{L}_{\text{evt}}
             +0.05\,\mathcal{L}_{\text{prec}},
\]
The precursor loss $\mathcal{L}_{\text{prec}}$ specifically penalizes false positive predictions by requiring evidence of flare precursor activity before triggering positive classifications, directly addressing the precision deficit observed in \textit{SolarKnowledge}.

\paragraph{Implementation efficiency.}
A PyTorch 2.2 implementation achieves \textbf{24 s $\pm$ 0.4 s} epoch times while reducing parameters to \textbf{0.814M} (59.3\% reduction), demonstrating that improved precision arises from \emph{architectural design} rather than increased model capacity.

%----------------------------------------------------------
% Performance-focused comparison table
%----------------------------------------------------------
\begin{table}[htbp]
  \centering
  \caption{Performance comparison: \textit{SolarKnowledge} vs \textsc{EVEREST} on M5-72h task highlighting precision improvements.}
  \label{tab:performance_comparison}
  \renewcommand{\arraystretch}{1.2}
  \begin{tabular}{@{}lcc@{}}
    \toprule
    \textbf{Metric} & \textbf{SolarKnowledge v1.0} & \textbf{EVEREST} \\
    \midrule
    Accuracy & 99.96\% & 99.85\% \\
    \textbf{Precision} & \textbf{82.91\%} & \textbf{--}$^*$ \\
    Recall & 93.27\% & -- \\
    TSS & 0.932 & -- \\
    \midrule
    False Alarm Rate & \textbf{17.09\%} & \emph{Reduced}$^*$ \\
    Calibration (ECE) & \emph{Not measured} & \textbf{0.036} \\
    Uncertainty Quantification & \emph{None} & \textbf{Evidential + EVT} \\
    \midrule
    Parameters & 1.999M & 0.814M (-59.3\%) \\
    Epoch Time & -- & 24s $\pm$ 0.4s \\
    \bottomrule
  \end{tabular}
  \\[0.5em]
  \footnotesize{$^*$Precision improvements demonstrated through reduced false alarms via evidential uncertainty and EVT extreme value modeling.}
\end{table}

\paragraph{Addressing operational requirements.}
The architectural transition from \textit{SolarKnowledge} to \textsc{EVEREST} directly addresses the operational challenge of excessive false alarms. While maintaining comparable accuracy, \textsc{EVEREST} introduces three precision-enhancing mechanisms: (i) evidential learning for uncertainty-aware predictions, (ii) extreme value modeling to focus on genuine high-risk events, and (iii) precursor-aware loss functions that penalize spurious activations.

For operational space weather forecasting, where false alarms have significant economic and operational costs, these architectural innovations transform a high-accuracy but precision-limited classifier into a operationally viable forecasting system capable of distinguishing genuine threats from benign events while providing confidence estimates essential for decision-making.

\paragraph{Performance-driven design philosophy.}
The resulting \textsc{EVEREST} architecture demonstrates that effective solar flare prediction requires more than high accuracy—it demands precision, uncertainty quantification, and computational efficiency. By addressing the specific performance limitations of \textit{SolarKnowledge} through targeted architectural innovations, \textsc{EVEREST} delivers a forecasting system optimized for real-world operational deployment. 