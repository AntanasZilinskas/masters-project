\documentclass[12pt,a4paper,oneside]{report}

%----------------------------------------------------------------------------------------
%   PACKAGES & BASIC SETUP
%----------------------------------------------------------------------------------------
\usepackage[utf8]{inputenc}
\usepackage[T1]{fontenc}
\usepackage[english]{babel}
\usepackage{microtype}
\usepackage{setspace}
\onehalfspacing
\usepackage{listings}

\usepackage[margin=1in]{geometry}
\setlength{\parindent}{0em}
\setlength{\parskip}{0.5em}

\usepackage{amsmath,amssymb,amsfonts}
\usepackage{graphicx}
\usepackage{booktabs}
\usepackage{subcaption}
\usepackage{enumitem}
\usepackage{titlesec}
\usepackage{longtable}

%----------------------------------------------------------------------------------------
%   HARVARD (AUTHOR-DATE) REFERENCING
%----------------------------------------------------------------------------------------
\usepackage[round,authoryear]{natbib}
% 'round' gives (Author, Year); 'authoryear' ensures the author-date format.
\bibliographystyle{agsm}  % Use Harvard-like style for the bibliography.

% Custom macros for consistent referencing
\newcommand{\citeauthorstyle}[1]{\citeauthor{#1} (\citeyear{#1})}  % Author (Year)
\newcommand{\citeparensstyle}[1]{(\citeauthor{#1}, \citeyear{#1})} % (Author, Year)

%----------------------------------------------------------------------------------------
%   HYPERLINKS (Load Last)
%----------------------------------------------------------------------------------------
\usepackage[colorlinks=true, linkcolor=blue, urlcolor=blue, citecolor=blue]{hyperref}

%----------------------------------------------------------------------------------------
%   GLOSSARIES PACKAGE FOR ABBREVIATIONS
%----------------------------------------------------------------------------------------
\usepackage[acronym]{glossaries}

% Define abbreviations
\newacronym{twod}{2D}{two dimensional}

%----------------------------------------------------------------------------------------
%   TITLE AND AUTHOR COMMANDS
%----------------------------------------------------------------------------------------
\newcommand{\reporttitle}{Solar Flare Prediction with Multimodal Transformer Architectures: A Novel Approach to Interpretable Space Weather Forecasting}
\newcommand{\reportauthor}{Antanas Žilinskas}
\newcommand{\reporttype}{Design Engineering Master’s Project}

% Adjust ToC Depth if desired
\setcounter{secnumdepth}{3}
\setcounter{tocdepth}{2}

\begin{document}

%----------------------------------------------------------------------------------------
%   TITLE PAGE
%----------------------------------------------------------------------------------------
%TC:ignore
\begin{titlepage}
\begin{center}

\vspace*{1.5cm}

%------------------------------------------------
% Logo
%------------------------------------------------
\includegraphics[width=4cm]{./figures/Imperial_College_London_new_logo}\\[1cm]

%------------------------------------------------
% Institution and Department
%------------------------------------------------
{\scshape\Large Imperial College London}\\[0.3cm]
{\Large Dyson School of Design Engineering}\\[1.5cm]

%------------------------------------------------
% Project Type
%------------------------------------------------
{\scshape\LARGE Design Engineering Master’s Project}\\[0.5cm]
{\Large (Early Stage Gateway Submission)}\\[0.5cm]

%------------------------------------------------
% Title
%------------------------------------------------
\rule{\linewidth}{0.5mm}\\[0.2cm] % Horizontal line and spacing

\begin{center}
{\Large \bfseries Solar Flare Prediction with Multimodal \\[0.1cm]
Transformer Architectures: \\[0.1cm]
A Novel Approach to Interpretable \\[0.5cm]
Space Weather Forecasting}
\end{center}

\rule{\linewidth}{0.5mm}\\[0.5cm] % Horizontal line and spacing
%------------------------------------------------
% Author and Supervisor
%------------------------------------------------
{\Large \textit{Author:}}\\[0.2cm]
{\Large \reportauthor}\\[0.5cm]

{\Large \textit{Supervised by:}}\\[0.2cm]
{\Large Professor Robert N Shorten}\\[0.5cm]

%------------------------------------------------
% Purpose / Statement (Formal)
%------------------------------------------------
{\normalsize This Early Stage Gateway submission is presented in partial fulfillment of the}\\[0.2cm]
{\normalsize requirements for the Design Engineering Master’s Project, part of the MEng}\\[0.2cm]
{\normalsize degree in Design Engineering.}\\[1.0cm]

%------------------------------------------------
% Date and Location
%------------------------------------------------
{\normalsize \today}\\[0.2cm]
{\normalsize London, United Kingdom}

\vfill

\end{center}
\end{titlepage}
%TC:endignore

\clearpage
\pagenumbering{roman}

%----------------------------------------------------------------------------------------
%   DECLARATION OF ORIGINALITY & COPYRIGHT DECLARATION
%----------------------------------------------------------------------------------------
%TC:ignore
\clearpage
\section*{Declaration of Originality}

\noindent
I hereby declare that the work presented in this thesis is my own, except where explicitly stated otherwise. To the best of my knowledge, it is original, and any collaborative contributions or referenced ideas have been duly acknowledged. Large Language Models (e.g., Claude and ChatGPT) were utilized to aid with project development, enhance the quality of writing, and as tools during research.

\vspace{2cm} % Space between the two declarations

\section*{Copyright Declaration}

\noindent
The copyright of this thesis rests with the author and is made available under a Creative Commons Attribution Non-Commercial No Derivatives licence. Researchers are free to copy, distribute, or transmit the thesis on the condition that they attribute it, that they do not use it for commercial purposes, and that they do not alter, transform, or build upon it. For any reuse or redistribution, researchers must make clear to others the licence terms of this work.
%TC:endignore

%----------------------------------------------------------------------------------------
%   ABSTRACT
%----------------------------------------------------------------------------------------
%TC:ignore
\clearpage
\section*{Abstract}
Solar flare forecasting is a critical challenge at the intersection of space weather prediction and advanced computational modeling. Current state-of-the-art approaches, including statistical methods, traditional machine learning models \cite{RefWorks:RefID:12-zheng2023comparative}, and deep learning architectures based on convolutional and recurrent networks \cite{RefWorks:RefID:10-jiao2020solar}, provide incremental improvements but often struggle with the intrinsic complexity, multimodality, and nonlinearity of solar data. These shortcomings restrict both predictive accuracy and lead times, especially for the rare but highly impactful X-class flares.

The following research propose a plan for a novel transformer-based methodology inspired by Large Language Models (LLMs) \cite{RefWorks:RefID:3-vaswani2023provided,RefWorks:RefID:2-abduallah2023operational} to address these limitations. By integrating spatial magnetogram imagery, time-series flux measurements, and related solar observations into a unified, self-attention-driven framework \cite{RefWorks:RefID:35-licllmate:}, approach aims to set new performance benchmarks. Research will employ self-supervised pretraining to learn generalizable representations of solar phenomena \cite{RefWorks:RefID:29-hoffmanntraining}, thereby overcoming data scarcity and improving model robustness. The resulting architecture seeks not only to enhance predictive skill but also to offer interpretability through attention mechanisms \cite{RefWorks:RefID:17-devlinbert:}, granting insights into which features and temporal intervals are most indicative of impending events.

The anticipated contributions include more accurate, early detection of major solar flares, improved understanding of their underlying drivers, and guidance for space weather stakeholders to better protect infrastructure and human activity dependent on reliable technological systems. Ultimately, this work aspires to establish a new standard for solar flare predictions using transformer-based architecture.
\noindent
%TC:endignore

%----------------------------------------------------------------------------------------
%   TABLE OF CONTENTS
%----------------------------------------------------------------------------------------
%TC:ignore
\clearpage
\tableofcontents
%TC:endignore

%----------------------------------------------------------------------------------------
%   ABBREVIATIONS, ACRONYMS, AND INITIALISMS
%----------------------------------------------------------------------------------------
% Uncomment if needed and exclude from word count
% %TC:ignore
% \clearpage
% \section*{Abbreviations, Acronyms, and Initialisms}
% \printglossary[type=\acronymtype, title={}]
% %TC:endignore

%----------------------------------------------------------------------------------------
%   FIGURES AND TABLES (OPTIONAL)
%----------------------------------------------------------------------------------------
% Uncomment if you have figures or tables and exclude from word count
% %TC:ignore
% \clearpage
% \listoffigures
% \clearpage
% \listoftables
% \clearpage
% %TC:endignore

\pagenumbering{arabic}
\setcounter{page}{1}

%----------------------------------------------------------------------------------------
%   CHAPTERS
%----------------------------------------------------------------------------------------
\chapter{Introduction}
\label{chap:intro}
Solar flares are colossal eruptions on the Sun’s surface, releasing immense energy across the electromagnetic spectrum and causing severe disturbances in Earth’s space weather environment \citeparensstyle{RefWorks:RefID:11-esa}. Even moderate flares can disrupt aviation, maritime communications, and satellite operations, while intense X-class events threaten power grids and crewed space missions \citeparensstyle{RefWorks:RefID:6-omatolaimpacts}. Table~\ref{tab:solar_flare_impacts} summarizes flare classes and their terrestrial impacts. Although precautionary measures exist, timely and accurate forecasting remains critical for allowing operators and agencies to prepare effectively.

%TC:ignore
\begin{table}[h!]
\centering
\caption{Solar Flare Classifications and Potential Terrestrial Impacts}
\label{tab:solar_flare_impacts}
\begin{tabular}{|c|c|p{8cm}|}
\hline
\textbf{Class} & \textbf{Peak X-ray Flux (W/m\textsuperscript{2})} & \textbf{Potential Impacts on Earth} \\ \hline
A              & $I < 10^{-8}$                                     & Minimal to no observable impact. \\ \hline
B              & $10^{-8} \leq I < 10^{-7}$                        & Generally negligible effects; minor disturbances in Earth's magnetosphere. \\ \hline
C              & $10^{-7} \leq I < 10^{-6}$                        & Small; few noticeable consequences on Earth. \\ \hline
M              & $10^{-6} \leq I < 10^{-5}$                        & Medium-sized; can cause brief radio blackouts affecting Earth's polar regions. Minor radiation storms sometimes follow. \\ \hline
X              & $I \geq 10^{-5}$                                  & Major events that can trigger planet-wide radio blackouts and long-lasting radiation storms. \\ \hline
\end{tabular}
\end{table}
%TC:endignore

Traditional statistical and early machine learning models (e.g., SVMs, random forests) achieved incremental improvements but failed to capture the complex temporal-spatial dynamics driving flares \citep{RefWorks:RefID:7-schrijver2009driving, RefWorks:RefID:12-zheng2023comparative}. Recent deep learning methods, employing LSTMs and CNNs, have enhanced performance by integrating temporal and spatial features, while attention mechanisms have further boosted mid-level flare predictions \citep{RefWorks:RefID:10-jiao2020solar, RefWorks:RefID:2-abduallah2023operational}. Yet, accurately forecasting rare, high-impact X-class flares and generalizing across solar cycles remain unresolved challenges \citep{RefWorks:RefID:14-hayes2021solar}.

To overcome these limitations, this research introduces a multimodal transformer architecture inspired by Large Language Model (LLM) principles \citep{RefWorks:RefID:3-vaswani2023provided}. Unlike traditional CNN-LSTM hybrids, it fuses diverse data—magnetograms, flux time series, and historical flare records—into one predictive pipeline \citep{RefWorks:RefID:2-abduallah2023operational, RefWorks:RefID:30-schmude2024prithvi}. By doing so, it aims to surpass current benchmarks, improve lead times, enhance interpretability, and establish a new standard for solar flare forecasting.

\chapter{Literature Review and Contextual Investigation}
\label{chap:literature-review-and-contextual-investigation}
Solar flare prediction has a long history of methodological evolution, given the growing complexity of our understanding of solar physics and the growing importance of timely forecasts. Early efforts primarily used statistical approaches, grounded in empirically derived heuristics and morphological classifications of solar active regions \citep{RefWorks:RefID:7-schrijver2009driving}. Though these statistical methods offered initial baselines, they were typically constrained by simplistic assumptions. As a result, they struggled to capture the inherently nonlinear and dynamic nature of the solar magnetic environment. Subsequent attempts to refine forecasts introduced traditional machine learning algorithms, such as Support Vector Machines (SVMs) and Random Forests (RFs), utilizing engineered features extracted from observed solar magnetic field parameters. While these techniques improved classification accuracy and provided more flexible decision boundaries than pure statistical models \citep{RefWorks:RefID:12-zheng2023comparative, RefWorks:RefID:2-abduallah2023operational}, they still faced limitations. Notably, such models often failed to adequately capture the temporal dimension critical to flare onset and lacked mechanisms for assimilating the diverse sensor modalities available.

The advancement of deep learning techniques furthered the effectiveness of solar flare prediction. Convolutional Neural Networks (CNNs) enabled the automated extraction of spatial features from high-resolution solar imagery, whereas Long Short-Term Memory (LSTM) networks could model time dependencies more effectively. By combining CNNs and LSTMs, researchers created hybrid architectures that jointly processed spatial magnetogram data and temporal flux measurements. This fusion yielded improvements over traditional ML models, particularly in differentiating higher-class flares (e.g., M-class and X-class) from weaker events \citep{RefWorks:RefID:2-abduallah2023operational, RefWorks:RefID:10-jiao2020solar}. Moreover, the introduction of attention mechanisms into LSTM-based models began to address the challenge of focusing on the most relevant features at critical time steps \citep{RefWorks:RefID:12-zheng2023comparative}. Given the complexity of the data available, research has also attempted to predict solar flare activity based on data of the Space-weather HMI Active Region Patches (SHARP) parameters \citep{RefWorks:RefID:2-abduallah2023operational}. These measures, derived from observational data, describe the magnetic field configuration and complexity. However, such abstractions lose out on the resolution needed to deliver predictions more reliably. More widely, despite these advancements, even state-of-the-art deep learning models have struggled with data scarcity for rare X-class flares and have shown limited ability to generalize across multiple solar cycles \citep{RefWorks:RefID:14-hayes2021solar}.

In parallel, the broader field of machine learning has advanced significantly with the introduction of transformer architectures \citep{RefWorks:RefID:3-vaswani2023provided}. Originally designed for natural language processing tasks, transformers leverage the concept of self-attention to model long-range dependencies without relying on recurrent operations. By doing so, they offer flexible contextual modeling, scalability, and an ability to consider global relationships within a sequence. Large Language Models (LLMs), such as GPT and BERT, have demonstrated the raw power of transformers in capturing nuanced patterns from considerable text corpora \citep{RefWorks:RefID:17-devlinbert:}. While these models were conceived for language, the underlying principles have proven remarkably transferable to other domains that benefit from long-context attention and rely on time-series data.

Outside of solar physics, transformers have shown promise in time-series forecasting tasks, climate modeling, and various scientific challenges where complex, nonlinear dynamics span multiple temporal and spatial scales. For instance, researchers have applied transformers to predict seasonal weather patterns, learn from satellite-derived Earth observation data, and analyze energy consumption trends—scenarios that, like solar activity, involve intricate relationships across time and space \citep{RefWorks:RefID:1-gettelman1997future, RefWorks:RefID:13-yıldız2023effect}. In climate modeling, transformers have begun to tackle the complexity of Earth system processes, potentially offering improved long-horizon forecasts over previous recurrent or convolutional models \citep{RefWorks:RefID:26-liu2024datasets, RefWorks:RefID:24-blumenfeldnasa}. In financial and economic time-series forecasting, transformers have demonstrated improved predictive performance and interpretability in understanding market shifts, which often require capturing long-term correlations \citep{RefWorks:RefID:30-schmude2024prithvi}. These successes suggest that transformer architectures could similarly deliver improvements in solar flare prediction.

When examining the existing body of solar flare forecasting literature, a clear gap emerges: very few studies have seriously considered transformer-based models, let alone experimented with self-supervised learning or multimodal fusion strategies tailored for this domain. While recent attempts have begun to incorporate attention-like mechanisms or even basic transformer variants into the pipeline \citep{RefWorks:RefID:2-abduallah2023operational}, they remain preliminary and have not fully leveraged the power of LLM-inspired architectures. There is sparse exploration into truly multimodal approaches that can fuse heterogeneous inputs—ranging from image-based magnetograms, line-of-sight magnetic field measurements, X-ray flux time series, to derived solar features—within a single, unified transformer framework. Equally rare is any detailed investigation of self-supervised pretraining strategies, which could enable the model to learn robust, general representations of solar activity before downstream supervised forecasting tasks.

This avenue of research could benefit the solar flare predictions community significantly. Solar activity spans multiple timescales, from minutes to days, and its driving processes involve complex spatial structures and evolving magnetic fields. Transformers, with their global attention, can potentially model these long-term dependencies far better than classical recurrent or convolutional architectures \citep{RefWorks:RefID:25-kaplanscaling}. Additionally, by treating input modalities as tokens within a transformer encoder, we can theoretically integrate images, scalar time series, and other data types in a more principled manner than ad-hoc model fusion strategies \citep{RefWorks:RefID:33-francisco2024multimodal}.

Beyond the modeling capabilities, another challenge in solar flare forecasting is the interpretability of predictions. Existing black-box deep learning models often yield predictions without illuminating what aspects of solar activity led to a forecasted flare. Transformers offer a unique interpretability avenue: attention weights. By examining attention maps, it may be possible to identify specific spatial regions or temporal intervals that the model deems critical, thereby providing the solar physics community with valuable insights into the underlying solar processes that precede flares \citep{RefWorks:RefID:34-gallagheri.}.

Yet, to realize these benefits, several challenges must be addressed. Data complexity is significant: solar data come in multiple formats (e.g., magnetograms, UV/EUV imagery, spectral flux), each requiring different preprocessing steps. Moreover, rare but high-impact flare classes (X-class) have limited examples, calling for domain adaptation, transfer learning, or synthetic data generation. Ensuring that the model can capture long-term dependencies extending over several days of solar observations is another nontrivial endeavor. Transformers are well-equipped for long sequences, but efficiency and careful architectural design remain paramount \citep{RefWorks:RefID:21-fedus2022switch}. Lastly, interpretability and uncertainty quantification are essential for operational trust—no forecasting system can be reliably used by mission operators or infrastructure managers without understanding the confidence and rationale behind its predictions.

The case for transformer-based methods in solar flare forecasting, therefore, rests firmly on these points. By using self-attention, transformers can fuse diverse signals that conventional architectures handle only in a piecemeal manner. The advanced representational power from self-supervised pretraining could help overcome data scarcity, learning general solar context before specializing in flare prediction. In doing so, these models can push beyond incremental gains, potentially offering a step-change in predictive skill. From earlier detection of imminent X-class flares to increased model interpretability and more robust generalization across solar cycles, a transformer-centered approach is well positioned to reshape the field.

\chapter{Project Objectives and scope}
\label{chap:project-objectives-and-scope}
\subsection{Objectives}
This project aims to set a new standard for solar flare forecasting by employing transformer-based, multimodal models. The primary objectives are to:

\begin{enumerate}
\item Develop a transformer architecture that integrates solar imagery, magnetograms, and X-ray flux time series into a unified predictive framework \cite{RefWorks:RefID:2-abduallah2023operational, RefWorks:RefID:12-zheng2023comparative}.
\item Explore self-supervised strategies to learn generalizable solar representations, addressing data scarcity and improving model robustness \cite{RefWorks:RefID:21-fedus2022switch, RefWorks:RefID:29-hoffmanntraining}.
\item Surpass current benchmarks on key solar forecasting metrics (TSS, POD, FAR), especially for rare, high-impact X-class flares \cite{RefWorks:RefID:2-abduallah2023operational, RefWorks:RefID:14-hayes2021solar}.
\end{enumerate}

In addition to these primary aims, secondary objectives include exploring interpretability tools, by visualizing attention maps to highlight key regions or temporal intervals \cite{RefWorks:RefID:2-abduallah2023operational}, implementing uncertainty quantification to inform operational risk assessments \cite{RefWorks:RefID:2-abduallah2023operational}, and incorporating domain-informed architectural modifications, such as tailored positional embeddings aligned with solar rotational periods \cite{RefWorks:RefID:15-harra2016characteristics, RefWorks:RefID:7-schrijver2009driving}.

\subsection{Scope and Constraints}
This work focuses on publicly available data from sources like SDO and GOES \cite{RefWorks:RefID:13-yıldız2023effect, RefWorks:RefID:6-omatolaimpacts}. Emphasis will be placed on M- and X-class flare prediction over at least one full solar cycle. While the methodology can extend to other datasets or incorporate additional physical processes, this study confines itself to well-established datasets and core predictive tasks \cite{RefWorks:RefID:1-gettelman1997future}, deferring more complex scenarios for future investigations.


\chapter{Methodological Approach}
\label{chap:methodological-approach}
\subsection{Data Selection and Preprocessing}
This project will leverage GOES X-ray flux time series for flare labeling, complemented by Solar Dynamics Observatory (SDO) imagery (HMI data) for spatial patterns and SHARP parameters representing magnetic field properties \cite{RefWorks:RefID:2-abduallah2023operational, RefWorks:RefID:12-zheng2023comparative}. Preprocessing involves normalization, resolution adjustments, temporal synchronization, and careful handling of missing values. Multiple years of data will be used, covering diverse solar conditions \cite{RefWorks:RefID:13-yıldız2023effect}.

\subsection{Modeling Approach}
The core methodology centers on a multimodal transformer pipeline. Research will start by training a baseline time-series transformer on X-ray flux data \cite{RefWorks:RefID:3-vaswani2023provided}. Next, self-supervised pretraining tasks (masked token modeling, temporal patch prediction) will learn general solar feature representations prior to supervised fine-tuning \cite{RefWorks:RefID:21-fedus2022switch, RefWorks:RefID:29-hoffmanntraining}. A Vision Transformer (ViT)-inspired component will process imagery, while a separate transformer encoder handles flux or magnetic inputs. Cross-attention layers then fuse these data streams into a unified predictive head \cite{RefWorks:RefID:2-abduallah2023operational}.

\subsection{Benchmarking and Validation Strategy}
Performance will be compared to established baselines—such as CNN-LSTM hybrids and simpler transformers—using metrics like the True Skill Statistic (TSS), Probability of Detection (POD), False Alarm Rate (FAR), and Heidke Skill Score (HSS) \cite{RefWorks:RefID:12-zheng2023comparative}. K-fold cross-validation across multiple solar cycles ensures robust generalization. Early evaluations will guide iterative refinements \cite{RefWorks:RefID:13-yıldız2023effect}.

\subsection{Feasibility and Tools}
The implementation will use PyTorch and run on high-performance local computing resources or GPU clusters. Adapting existing transformer codebases from climate and Earth observation domains will streamline development, ensuring both reliability and efficient experimentation \cite{RefWorks:RefID:3-vaswani2023provided, RefWorks:RefID:1-gettelman1997future, RefWorks:RefID:30-schmude2024prithvi}.

\chapter{Project Management and Timeline}
\label{chap:project-management-and-timeline}
\subsection{Work Plan}
Term 1 established the project’s theoretical foundation via an extensive literature review, clarifying research gaps and guiding the methodological approach \cite{RefWorks:RefID:27-minaeelarge}. In parallel, initial technical setups and a preliminary experiment—training a small language model on a Shakespearean corpus—validated the computational workflow and transformer routines (see Listing~\ref{lst:sample_output}), whilst providing a great breakdown of "Attention Is All You Need" paper \cite{RefWorks:RefID:3-vaswani2023provided}.

%TC:ignore
\begin{lstlisting}[caption={Sample Output from the Preliminary Language Model}, label={lst:sample_output}]
MARCIUS:
God less unfound
A harm upon it!

Lord Marcius 'sut doth the twenty world,
And flover high gives, she'll hear shall mine mothes,
Having been most post were soul's a conspiror
And to Rome well thricke.
I would you drive you me to it.

SLEONTAS:
Take it to sign answer.

LADY ANNE:
Yea, bid me you upon that subject come to have hath
The prepartuative you.

GLOUCESTER:

GLOUCESTER:
You had not to stand of, when measure.
What Lady? this man! Catesby Is to think:
Yout the varlet and day not
\end{lstlisting}
%TC:endignore

\subsection{Term 2 and Term 3}
Term 2 will implement full-scale transformer models tailored to solar data \cite{RefWorks:RefID:33-francisco2024multimodal}, investigate self-supervised pretraining, and integrate multiple input modalities. Iterative refinement, benchmarking against baselines \cite{RefWorks:RefID:30-schmude2024prithvi}, and architectural improvements will be central activities. Term 3 will focus on robust validation, including cross-validation, interpretability analyses (via attention maps) \cite{RefWorks:RefID:29-hoffmanntraining}, uncertainty quantification, and final performance reporting. Deliverables include a refined predictive model, comprehensive evaluations, and operational recommendations \cite{RefWorks:RefID:35-licllmate:}.

\subsection{Week-by-week breakdown}
\begin{longtable}{|p{2cm}|p{13cm}|}
\hline
\textbf{Week} & \textbf{Tasks and Milestones} \\
\hline
\multicolumn{2}{|c|}{\textbf{Term 2 (Spring Term)}} \\
\hline
Week 1 (Jan 8 - Jan 14) & Finalize self-supervised pretraining tasks. Prepare datasets for multimodal input integration, ensuring alignment and preprocessing of magnetogram images, time-series flux, and SHARP parameters. \\
\hline
Week 2 (Jan 15 - Jan 21) & Implement the baseline transformer architecture for time-series flux data. Train and validate the initial model on a reduced dataset to debug computational workflow. \\
\hline
Week 3 (Jan 22 - Jan 28) & Introduce multimodal input handling by integrating spatial magnetograms and temporal flux measurements into the model pipeline. Perform initial tests with simplified inputs. \\
\hline
Week 4 (Jan 29 - Feb 4) & Develop and implement cross-attention layers for fusing diverse modalities. Conduct preliminary training on multimodal datasets. \\
\hline
Week 5 (Feb 5 - Feb 11) & Conduct benchmarking of the current model against CNN-LSTM and other baseline architectures using True Skill Statistics (TSS) and other metrics. Refine architecture based on performance gaps. \\
\hline
Week 6 (Feb 12 - Feb 18) & Focus on interpretability: implement attention map visualization and begin analyzing which features contribute most to flare predictions. Document insights. \\
\hline
Week 7 (Feb 19 - Feb 25) & Explore uncertainty quantification methods. Implement and test uncertainty-aware predictions to evaluate model robustness. \\
\hline
Week 8 (Feb 26 - Mar 3) & Conduct a full round of validation using cross-validation over multiple solar cycles. Ensure the model generalizes well across different data distributions. \\
\hline
Week 9 (Mar 4 - Mar 10) & Prepare for Late Stage Gateway: finalize a comprehensive presentation of model design, current progress, challenges, and schematic work plan for the remaining project duration. \\
\hline
Week 10 (Mar 11 - Mar 17) & Final rehearsal of the Late Stage Gateway presentation. Collect and address feedback from peers and supervisors. Prepare visual aids such as attention maps and model outputs for the presentation. \\
\hline
Week 11 (Mar 18 - Mar 21) & Deliver the Late Stage Gateway presentation. Document feedback and create an updated work plan for the final stages of the project leading to June 5 submission. \\
\hline
\multicolumn{2}{|c|}{\textbf{Term 3 (Summer Term)}} \\
\hline
Week 1 (Mar 31 - Apr 6) & Incorporate feedback from the Late Stage Gateway into project plans. Begin final refinements of the multimodal transformer model based on validation insights. \\
\hline
Week 2 (Apr 7 - Apr 13) & Conduct additional experiments to validate improvements and robustness. Begin writing the introduction, background, and methodology sections of the final report. \\
\hline
Week 3 (Apr 14 - Apr 20) & Finalize interpretability tools (e.g., attention maps) and incorporate into analysis. Draft results section with a focus on benchmarking against baseline methods. \\
\hline
Week 4 (Apr 21 - Apr 27) & Integrate social, environmental, and technological impact analyses into the discussion section. Update diagrams and visualizations for the final report. \\
\hline
Week 5 (Apr 28 - May 4) & Review the full report draft with peers and supervisors. Incorporate feedback into final report revisions. \\
\hline
Week 6 (May 5 - May 11) & Conduct final testing and validation to ensure all criteria are met. Refine results and discussion sections based on testing insights. \\
\hline
Week 7 (May 12 - May 18) & Complete final report draft and submit for supervisor review. Prepare a slide deck for Demo Day and begin rehearsals. \\
\hline
Week 8 (May 19 - May 25) & Polish and finalize the project report. Conduct rehearsals for Viva and Demo Day presentations. \\
\hline
Week 9 (May 26 - Jun 1) & Submit the final report (Deadline: June 5). Finalize presentation materials for Demo Day. Prepare technical documentation and codebase for demonstration. \\
\hline
Week 10 (Jun 2 - Jun 8) & Deliver Demo Day presentation (Scheduled for the week of June 9). Prepare for VIVA sessions with practice Q and A sessions. \\
\hline
Week 11 (Jun 9 - Jun 15) & Deliver Viva presentation and Q and A session (Scheduled for the week of June 16). Document final reflections and insights for personal learning and growth. \\
\hline
\end{longtable}

\subsection{Risks and Mitigation}
Potential risks such as limited training data or computational bottlenecks have been considered from the outset \cite{RefWorks:RefID:22-gao2020pile:}. Mitigation strategies—ranging from alternative datasets \cite{RefWorks:RefID:26-liu2024datasets} to more efficient transformer variants \cite{RefWorks:RefID:21-fedus2022switch}—are in place, ensuring that the project’s momentum will not be unduly hindered \cite{RefWorks:RefID:18-rajbhandarizero:}.


\chapter{Expected Contributions and Impact}
\label{chap:expected-contributions-and-impact}
This research aims to establish a standard for solar flare forecasting by introducing transformer-based \cite{RefWorks:RefID:3-vaswani2023provided,RefWorks:RefID:29-hoffmanntraining}, multimodal \cite{RefWorks:RefID:30-schmude2024prithvi}, and self-supervised learning strategies specifically tailored to complex solar data. By integrating diverse data types—spatial magnetograms, flux time series, and potentially additional solar observations—into a unified framework \cite{RefWorks:RefID:33-francisco2024multimodal,RefWorks:RefID:2-abduallah2023operational}, this project can yield richer, more accurate forecasts with longer lead times and enhanced interpretability \cite{RefWorks:RefID:35-licllmate:}.

\chapter{Conclusion and Next Steps}
\label{chap:conclusion-and-next-steps}
In conclusion, this project aligns latest transformer architectures with the pressing need for more accurate, interpretable solar flare forecasts. By systematically addressing the challenges of data complexity, rare events, and long-range temporal dependencies, it aims to push the boundaries of current methodologies and operational capabilities.

Moving forward, the immediate next steps involve finalizing the model design, initiating early experiments with multimodal transformers, and iteratively refining approaches based on preliminary results. Over subsequent terms, this iterative process will guide the transformation of theoretical concepts into a robust, high-impact predictive framework.

%----------------------------------------------------------------------------------------
%   REFERENCES
%----------------------------------------------------------------------------------------
%TC:ignore
\clearpage
\bibliographystyle{agsm}  % Harvard-like style
\bibliography{References}
%TC:endignore

\end{document}
