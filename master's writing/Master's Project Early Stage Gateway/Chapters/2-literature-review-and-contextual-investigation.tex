Solar flare prediction has a long history of methodological evolution, given the growing complexity of our understanding of solar physics and the growing importance of timely forecasts. Early efforts primarily used statistical approaches, grounded in empirically derived heuristics and morphological classifications of solar active regions \citep{RefWorks:RefID:7-schrijver2009driving}. Though these statistical methods offered initial baselines, they were typically constrained by simplistic assumptions. As a result, they struggled to capture the inherently nonlinear and dynamic nature of the solar magnetic environment. Subsequent attempts to refine forecasts introduced traditional machine learning algorithms, such as Support Vector Machines (SVMs) and Random Forests (RFs), utilizing engineered features extracted from observed solar magnetic field parameters. While these techniques improved classification accuracy and provided more flexible decision boundaries than pure statistical models \citep{RefWorks:RefID:12-zheng2023comparative, RefWorks:RefID:2-abduallah2023operational}, they still faced limitations. Notably, such models often failed to adequately capture the temporal dimension critical to flare onset and lacked mechanisms for assimilating the diverse sensor modalities available.

The advancement of deep learning techniques furthered the effectiveness of solar flare prediction. Convolutional Neural Networks (CNNs) enabled the automated extraction of spatial features from high-resolution solar imagery, whereas Long Short-Term Memory (LSTM) networks could model time dependencies more effectively. By combining CNNs and LSTMs, researchers created hybrid architectures that jointly processed spatial magnetogram data and temporal flux measurements. This fusion yielded improvements over traditional ML models, particularly in differentiating higher-class flares (e.g., M-class and X-class) from weaker events \citep{RefWorks:RefID:2-abduallah2023operational, RefWorks:RefID:10-jiao2020solar}. Moreover, the introduction of attention mechanisms into LSTM-based models began to address the challenge of focusing on the most relevant features at critical time steps \citep{RefWorks:RefID:12-zheng2023comparative}. Given the complexity of the data available, research has also attempted to predict solar flare activity based on data of the Space-weather HMI Active Region Patches (SHARP) parameters \citep{RefWorks:RefID:2-abduallah2023operational}. These measures, derived from observational data, describe the magnetic field configuration and complexity. However, such abstractions lose out on the resolution needed to deliver predictions more reliably. More widely, despite these advancements, even state-of-the-art deep learning models have struggled with data scarcity for rare X-class flares and have shown limited ability to generalize across multiple solar cycles \citep{RefWorks:RefID:14-hayes2021solar}.

In parallel, the broader field of machine learning has advanced significantly with the introduction of transformer architectures \citep{RefWorks:RefID:3-vaswani2023provided}. Originally designed for natural language processing tasks, transformers leverage the concept of self-attention to model long-range dependencies without relying on recurrent operations. By doing so, they offer flexible contextual modeling, scalability, and an ability to consider global relationships within a sequence. Large Language Models (LLMs), such as GPT and BERT, have demonstrated the raw power of transformers in capturing nuanced patterns from considerable text corpora \citep{RefWorks:RefID:17-devlinbert:}. While these models were conceived for language, the underlying principles have proven remarkably transferable to other domains that benefit from long-context attention and rely on time-series data.

Outside of solar physics, transformers have shown promise in time-series forecasting tasks, climate modeling, and various scientific challenges where complex, nonlinear dynamics span multiple temporal and spatial scales. For instance, researchers have applied transformers to predict seasonal weather patterns, learn from satellite-derived Earth observation data, and analyze energy consumption trends—scenarios that, like solar activity, involve intricate relationships across time and space \citep{RefWorks:RefID:1-gettelman1997future, RefWorks:RefID:13-yıldız2023effect}. In climate modeling, transformers have begun to tackle the complexity of Earth system processes, potentially offering improved long-horizon forecasts over previous recurrent or convolutional models \citep{RefWorks:RefID:26-liu2024datasets, RefWorks:RefID:24-blumenfeldnasa}. In financial and economic time-series forecasting, transformers have demonstrated improved predictive performance and interpretability in understanding market shifts, which often require capturing long-term correlations \citep{RefWorks:RefID:30-schmude2024prithvi}. These successes suggest that transformer architectures could similarly deliver improvements in solar flare prediction.

When examining the existing body of solar flare forecasting literature, a clear gap emerges: very few studies have seriously considered transformer-based models, let alone experimented with self-supervised learning or multimodal fusion strategies tailored for this domain. While recent attempts have begun to incorporate attention-like mechanisms or even basic transformer variants into the pipeline \citep{RefWorks:RefID:2-abduallah2023operational}, they remain preliminary and have not fully leveraged the power of LLM-inspired architectures. There is sparse exploration into truly multimodal approaches that can fuse heterogeneous inputs—ranging from image-based magnetograms, line-of-sight magnetic field measurements, X-ray flux time series, to derived solar features—within a single, unified transformer framework. Equally rare is any detailed investigation of self-supervised pretraining strategies, which could enable the model to learn robust, general representations of solar activity before downstream supervised forecasting tasks.

This avenue of research could benefit the solar flare predictions community significantly. Solar activity spans multiple timescales, from minutes to days, and its driving processes involve complex spatial structures and evolving magnetic fields. Transformers, with their global attention, can potentially model these long-term dependencies far better than classical recurrent or convolutional architectures \citep{RefWorks:RefID:25-kaplanscaling}. Additionally, by treating input modalities as tokens within a transformer encoder, we can theoretically integrate images, scalar time series, and other data types in a more principled manner than ad-hoc model fusion strategies \citep{RefWorks:RefID:33-francisco2024multimodal}.

Beyond the modeling capabilities, another challenge in solar flare forecasting is the interpretability of predictions. Existing black-box deep learning models often yield predictions without illuminating what aspects of solar activity led to a forecasted flare. Transformers offer a unique interpretability avenue: attention weights. By examining attention maps, it may be possible to identify specific spatial regions or temporal intervals that the model deems critical, thereby providing the solar physics community with valuable insights into the underlying solar processes that precede flares \citep{RefWorks:RefID:34-gallagheri.}.

Yet, to realize these benefits, several challenges must be addressed. Data complexity is significant: solar data come in multiple formats (e.g., magnetograms, UV/EUV imagery, spectral flux), each requiring different preprocessing steps. Moreover, rare but high-impact flare classes (X-class) have limited examples, calling for domain adaptation, transfer learning, or synthetic data generation. Ensuring that the model can capture long-term dependencies extending over several days of solar observations is another nontrivial endeavor. Transformers are well-equipped for long sequences, but efficiency and careful architectural design remain paramount \citep{RefWorks:RefID:21-fedus2022switch}. Lastly, interpretability and uncertainty quantification are essential for operational trust—no forecasting system can be reliably used by mission operators or infrastructure managers without understanding the confidence and rationale behind its predictions.

The case for transformer-based methods in solar flare forecasting, therefore, rests firmly on these points. By using self-attention, transformers can fuse diverse signals that conventional architectures handle only in a piecemeal manner. The advanced representational power from self-supervised pretraining could help overcome data scarcity, learning general solar context before specializing in flare prediction. In doing so, these models can push beyond incremental gains, potentially offering a step-change in predictive skill. From earlier detection of imminent X-class flares to increased model interpretability and more robust generalization across solar cycles, a transformer-centered approach is well positioned to reshape the field.