\subsection{Data Selection and Preprocessing}
This project will leverage GOES X-ray flux time series for flare labeling, complemented by Solar Dynamics Observatory (SDO) imagery (HMI data) for spatial patterns and SHARP parameters representing magnetic field properties \cite{RefWorks:RefID:2-abduallah2023operational, RefWorks:RefID:12-zheng2023comparative}. Preprocessing involves normalization, resolution adjustments, temporal synchronization, and careful handling of missing values. Multiple years of data will be used, covering diverse solar conditions \cite{RefWorks:RefID:13-yıldız2023effect}.

\subsection{Modeling Approach}
The core methodology centers on a multimodal transformer pipeline. Research will start by training a baseline time-series transformer on X-ray flux data \cite{RefWorks:RefID:3-vaswani2023provided}. Next, self-supervised pretraining tasks (masked token modeling, temporal patch prediction) will learn general solar feature representations prior to supervised fine-tuning \cite{RefWorks:RefID:21-fedus2022switch, RefWorks:RefID:29-hoffmanntraining}. A Vision Transformer (ViT)-inspired component will process imagery, while a separate transformer encoder handles flux or magnetic inputs. Cross-attention layers then fuse these data streams into a unified predictive head \cite{RefWorks:RefID:2-abduallah2023operational}.

\subsection{Benchmarking and Validation Strategy}
Performance will be compared to established baselines—such as CNN-LSTM hybrids and simpler transformers—using metrics like the True Skill Statistic (TSS), Probability of Detection (POD), False Alarm Rate (FAR), and Heidke Skill Score (HSS) \cite{RefWorks:RefID:12-zheng2023comparative}. K-fold cross-validation across multiple solar cycles ensures robust generalization. Early evaluations will guide iterative refinements \cite{RefWorks:RefID:13-yıldız2023effect}.

\subsection{Feasibility and Tools}
The implementation will use PyTorch and run on high-performance local computing resources or GPU clusters. Adapting existing transformer codebases from climate and Earth observation domains will streamline development, ensuring both reliability and efficient experimentation \cite{RefWorks:RefID:3-vaswani2023provided, RefWorks:RefID:1-gettelman1997future, RefWorks:RefID:30-schmude2024prithvi}.