\subsection{Objectives}
This project aims to set a new standard for solar flare forecasting by employing transformer-based, multimodal models. The primary objectives are to:

\begin{enumerate}
\item Develop a transformer architecture that integrates solar imagery, magnetograms, and X-ray flux time series into a unified predictive framework \cite{RefWorks:RefID:2-abduallah2023operational, RefWorks:RefID:12-zheng2023comparative}.
\item Explore self-supervised strategies to learn generalizable solar representations, addressing data scarcity and improving model robustness \cite{RefWorks:RefID:21-fedus2022switch, RefWorks:RefID:29-hoffmanntraining}.
\item Surpass current benchmarks on key solar forecasting metrics (TSS, POD, FAR), especially for rare, high-impact X-class flares \cite{RefWorks:RefID:2-abduallah2023operational, RefWorks:RefID:14-hayes2021solar}.
\end{enumerate}

In addition to these primary aims, secondary objectives include exploring interpretability tools, by visualizing attention maps to highlight key regions or temporal intervals \cite{RefWorks:RefID:2-abduallah2023operational}, implementing uncertainty quantification to inform operational risk assessments \cite{RefWorks:RefID:2-abduallah2023operational}, and incorporating domain-informed architectural modifications, such as tailored positional embeddings aligned with solar rotational periods \cite{RefWorks:RefID:15-harra2016characteristics, RefWorks:RefID:7-schrijver2009driving}.

\subsection{Scope and Constraints}
This work focuses on publicly available data from sources like SDO and GOES \cite{RefWorks:RefID:13-yıldız2023effect, RefWorks:RefID:6-omatolaimpacts}. Emphasis will be placed on M- and X-class flare prediction over at least one full solar cycle. While the methodology can extend to other datasets or incorporate additional physical processes, this study confines itself to well-established datasets and core predictive tasks \cite{RefWorks:RefID:1-gettelman1997future}, deferring more complex scenarios for future investigations.
