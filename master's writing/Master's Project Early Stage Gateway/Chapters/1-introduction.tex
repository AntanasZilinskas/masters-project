Solar flares are colossal eruptions on the Sun’s surface, releasing immense energy across the electromagnetic spectrum and causing severe disturbances in Earth’s space weather environment \citeparensstyle{RefWorks:RefID:11-esa}. Even moderate flares can disrupt aviation, maritime communications, and satellite operations, while intense X-class events threaten power grids and crewed space missions \citeparensstyle{RefWorks:RefID:6-omatolaimpacts}. Table~\ref{tab:solar_flare_impacts} summarizes flare classes and their terrestrial impacts. Although precautionary measures exist, timely and accurate forecasting remains critical for allowing operators and agencies to prepare effectively.

%TC:ignore
\begin{table}[h!]
\centering
\caption{Solar Flare Classifications and Potential Terrestrial Impacts}
\label{tab:solar_flare_impacts}
\begin{tabular}{|c|c|p{8cm}|}
\hline
\textbf{Class} & \textbf{Peak X-ray Flux (W/m\textsuperscript{2})} & \textbf{Potential Impacts on Earth} \\ \hline
A              & $I < 10^{-8}$                                     & Minimal to no observable impact. \\ \hline
B              & $10^{-8} \leq I < 10^{-7}$                        & Generally negligible effects; minor disturbances in Earth's magnetosphere. \\ \hline
C              & $10^{-7} \leq I < 10^{-6}$                        & Small; few noticeable consequences on Earth. \\ \hline
M              & $10^{-6} \leq I < 10^{-5}$                        & Medium-sized; can cause brief radio blackouts affecting Earth's polar regions. Minor radiation storms sometimes follow. \\ \hline
X              & $I \geq 10^{-5}$                                  & Major events that can trigger planet-wide radio blackouts and long-lasting radiation storms. \\ \hline
\end{tabular}
\end{table}
%TC:endignore

Traditional statistical and early machine learning models (e.g., SVMs, random forests) achieved incremental improvements but failed to capture the complex temporal-spatial dynamics driving flares \citep{RefWorks:RefID:7-schrijver2009driving, RefWorks:RefID:12-zheng2023comparative}. Recent deep learning methods, employing LSTMs and CNNs, have enhanced performance by integrating temporal and spatial features, while attention mechanisms have further boosted mid-level flare predictions \citep{RefWorks:RefID:10-jiao2020solar, RefWorks:RefID:2-abduallah2023operational}. Yet, accurately forecasting rare, high-impact X-class flares and generalizing across solar cycles remain unresolved challenges \citep{RefWorks:RefID:14-hayes2021solar}.

To overcome these limitations, this research introduces a multimodal transformer architecture inspired by Large Language Model (LLM) principles \citep{RefWorks:RefID:3-vaswani2023provided}. Unlike traditional CNN-LSTM hybrids, it fuses diverse data—magnetograms, flux time series, and historical flare records—into one predictive pipeline \citep{RefWorks:RefID:2-abduallah2023operational, RefWorks:RefID:30-schmude2024prithvi}. By doing so, it aims to surpass current benchmarks, improve lead times, enhance interpretability, and establish a new standard for solar flare forecasting.