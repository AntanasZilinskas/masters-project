\subsection{Work Plan}
Term 1 established the project’s theoretical foundation via an extensive literature review, clarifying research gaps and guiding the methodological approach \cite{RefWorks:RefID:27-minaeelarge}. In parallel, initial technical setups and a preliminary experiment—training a small language model on a Shakespearean corpus—validated the computational workflow and transformer routines (see Listing~\ref{lst:sample_output}), whilst providing a great breakdown of "Attention Is All You Need" paper \cite{RefWorks:RefID:3-vaswani2023provided}.

%TC:ignore
\begin{lstlisting}[caption={Sample Output from the Preliminary Language Model}, label={lst:sample_output}]
MARCIUS:
God less unfound
A harm upon it!

Lord Marcius 'sut doth the twenty world,
And flover high gives, she'll hear shall mine mothes,
Having been most post were soul's a conspiror
And to Rome well thricke.
I would you drive you me to it.

SLEONTAS:
Take it to sign answer.

LADY ANNE:
Yea, bid me you upon that subject come to have hath
The prepartuative you.

GLOUCESTER:

GLOUCESTER:
You had not to stand of, when measure.
What Lady? this man! Catesby Is to think:
Yout the varlet and day not
\end{lstlisting}
%TC:endignore

\subsection{Term 2 and Term 3}
Term 2 will implement full-scale transformer models tailored to solar data \cite{RefWorks:RefID:33-francisco2024multimodal}, investigate self-supervised pretraining, and integrate multiple input modalities. Iterative refinement, benchmarking against baselines \cite{RefWorks:RefID:30-schmude2024prithvi}, and architectural improvements will be central activities. Term 3 will focus on robust validation, including cross-validation, interpretability analyses (via attention maps) \cite{RefWorks:RefID:29-hoffmanntraining}, uncertainty quantification, and final performance reporting. Deliverables include a refined predictive model, comprehensive evaluations, and operational recommendations \cite{RefWorks:RefID:35-licllmate:}.

\subsection{Week-by-week breakdown}
\begin{longtable}{|p{2cm}|p{13cm}|}
\hline
\textbf{Week} & \textbf{Tasks and Milestones} \\
\hline
\multicolumn{2}{|c|}{\textbf{Term 2 (Spring Term)}} \\
\hline
Week 1 (Jan 8 - Jan 14) & Finalize self-supervised pretraining tasks. Prepare datasets for multimodal input integration, ensuring alignment and preprocessing of magnetogram images, time-series flux, and SHARP parameters. \\
\hline
Week 2 (Jan 15 - Jan 21) & Implement the baseline transformer architecture for time-series flux data. Train and validate the initial model on a reduced dataset to debug computational workflow. \\
\hline
Week 3 (Jan 22 - Jan 28) & Introduce multimodal input handling by integrating spatial magnetograms and temporal flux measurements into the model pipeline. Perform initial tests with simplified inputs. \\
\hline
Week 4 (Jan 29 - Feb 4) & Develop and implement cross-attention layers for fusing diverse modalities. Conduct preliminary training on multimodal datasets. \\
\hline
Week 5 (Feb 5 - Feb 11) & Conduct benchmarking of the current model against CNN-LSTM and other baseline architectures using True Skill Statistics (TSS) and other metrics. Refine architecture based on performance gaps. \\
\hline
Week 6 (Feb 12 - Feb 18) & Focus on interpretability: implement attention map visualization and begin analyzing which features contribute most to flare predictions. Document insights. \\
\hline
Week 7 (Feb 19 - Feb 25) & Explore uncertainty quantification methods. Implement and test uncertainty-aware predictions to evaluate model robustness. \\
\hline
Week 8 (Feb 26 - Mar 3) & Conduct a full round of validation using cross-validation over multiple solar cycles. Ensure the model generalizes well across different data distributions. \\
\hline
Week 9 (Mar 4 - Mar 10) & Prepare for Late Stage Gateway: finalize a comprehensive presentation of model design, current progress, challenges, and schematic work plan for the remaining project duration. \\
\hline
Week 10 (Mar 11 - Mar 17) & Final rehearsal of the Late Stage Gateway presentation. Collect and address feedback from peers and supervisors. Prepare visual aids such as attention maps and model outputs for the presentation. \\
\hline
Week 11 (Mar 18 - Mar 21) & Deliver the Late Stage Gateway presentation. Document feedback and create an updated work plan for the final stages of the project leading to June 5 submission. \\
\hline
\multicolumn{2}{|c|}{\textbf{Term 3 (Summer Term)}} \\
\hline
Week 1 (Mar 31 - Apr 6) & Incorporate feedback from the Late Stage Gateway into project plans. Begin final refinements of the multimodal transformer model based on validation insights. \\
\hline
Week 2 (Apr 7 - Apr 13) & Conduct additional experiments to validate improvements and robustness. Begin writing the introduction, background, and methodology sections of the final report. \\
\hline
Week 3 (Apr 14 - Apr 20) & Finalize interpretability tools (e.g., attention maps) and incorporate into analysis. Draft results section with a focus on benchmarking against baseline methods. \\
\hline
Week 4 (Apr 21 - Apr 27) & Integrate social, environmental, and technological impact analyses into the discussion section. Update diagrams and visualizations for the final report. \\
\hline
Week 5 (Apr 28 - May 4) & Review the full report draft with peers and supervisors. Incorporate feedback into final report revisions. \\
\hline
Week 6 (May 5 - May 11) & Conduct final testing and validation to ensure all criteria are met. Refine results and discussion sections based on testing insights. \\
\hline
Week 7 (May 12 - May 18) & Complete final report draft and submit for supervisor review. Prepare a slide deck for Demo Day and begin rehearsals. \\
\hline
Week 8 (May 19 - May 25) & Polish and finalize the project report. Conduct rehearsals for Viva and Demo Day presentations. \\
\hline
Week 9 (May 26 - Jun 1) & Submit the final report (Deadline: June 5). Finalize presentation materials for Demo Day. Prepare technical documentation and codebase for demonstration. \\
\hline
Week 10 (Jun 2 - Jun 8) & Deliver Demo Day presentation (Scheduled for the week of June 9). Prepare for VIVA sessions with practice Q and A sessions. \\
\hline
Week 11 (Jun 9 - Jun 15) & Deliver Viva presentation and Q and A session (Scheduled for the week of June 16). Document final reflections and insights for personal learning and growth. \\
\hline
\end{longtable}

\subsection{Risks and Mitigation}
Potential risks such as limited training data or computational bottlenecks have been considered from the outset \cite{RefWorks:RefID:22-gao2020pile:}. Mitigation strategies—ranging from alternative datasets \cite{RefWorks:RefID:26-liu2024datasets} to more efficient transformer variants \cite{RefWorks:RefID:21-fedus2022switch}—are in place, ensuring that the project’s momentum will not be unduly hindered \cite{RefWorks:RefID:18-rajbhandarizero:}.