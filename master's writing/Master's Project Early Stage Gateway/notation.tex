% quick way of adding a figure
\newcommand{\fig}[3]{
 \begin{center}
 \scalebox{#3}{\includegraphics[#2]{#1}}
 \end{center}
}

%\newcommand*{\point}[1]{\vec{\mkern0mu#1}}
\newcommand{\ci}[0]{\perp\!\!\!\!\!\perp} % conditional independence
\newcommand{\point}[1]{{#1}} % points 
\renewcommand{\vec}[1]{{\boldsymbol{{#1}}}} % vector
\newcommand{\mat}[1]{{\boldsymbol{{#1}}}} % matrix
\newcommand{\R}[0]{\mathds{R}} % real numbers
\newcommand{\Z}[0]{\mathds{Z}} % integers
\newcommand{\N}[0]{\mathds{N}} % natural numbers
\newcommand{\nat}[0]{\mathds{N}} % natural numbers
\newcommand{\Q}[0]{\mathds{Q}} % rational numbers
\ifxetex
\newcommand{\C}[0]{\mathds{C}} % complex numbers
\else
\newcommand{\C}[0]{\mathds{C}} % complex numbers
\fi
\newcommand{\tr}[0]{\text{tr}} % trace
\renewcommand{\d}[0]{\mathrm{d}} % total derivative
\newcommand{\inv}{^{-1}} % inverse
\newcommand{\id}{\mathrm{id}} % identity mapping
\renewcommand{\dim}{\mathrm{dim}} % dimension
\newcommand{\rank}[0]{\mathrm{rk}} % rank
\newcommand{\determ}[1]{\mathrm{det}(#1)} % determinant
\newcommand{\scp}[2]{\langle #1 , #2 \rangle}
\newcommand{\kernel}[0]{\mathrm{ker}} % kernel/nullspace
\newcommand{\img}[0]{\mathrm{Im}} % image
\newcommand{\idx}[1]{{(#1)}}
\DeclareMathOperator*{\diag}{diag}
\newcommand{\E}{\mathds{E}} % expectation
\newcommand{\var}{\mathds{V}} % variance
\newcommand{\gauss}[2]{\mathcal{N}\big(#1,\,#2\big)} % gaussian distribution N(.,.)
\newcommand{\gaussx}[3]{\mathcal{N}\big(#1\,|\,#2,\,#3\big)} % gaussian distribution N(.|.,.)
\newcommand{\gaussBig}[2]{\mathcal{N}\left(#1,\,#2\right)} % see above, but with brackets that adjust to the height of the arguments
\newcommand{\gaussxBig}[3]{\mathcal{N}\left(#1\,|\,#2,\,#3\right)} % see above, but with brackets that adjust to the height of the arguments
\DeclareMathOperator{\cov}{Cov} % covariance (matrix) 
\ifxetex
\renewcommand{\T}[0]{^\top} % transpose
\else
\newcommand{\T}[0]{^\top}
\fi
% matrix determinant
\newcommand{\matdet}[1]{
\left|
\begin{matrix}
#1
\end{matrix}
\right|
}



%%% various color definitions
\definecolor{darkgreen}{rgb}{0,0.6,0}

\newcommand{\blue}[1]{{\color{blue}#1}}
\newcommand{\red}[1]{{\color{red}#1}}
\newcommand{\green}[1]{{\color{darkgreen}#1}}
\newcommand{\orange}[1]{{\color{orange}#1}}
\newcommand{\magenta}[1]{{\color{magenta}#1}}
\newcommand{\cyan}[1]{{\color{cyan}#1}}


% redefine emph
\renewcommand{\emph}[1]{\blue{\bf{#1}}}

% place a colored box around a character
\gdef\colchar#1#2{%
  \tikz[baseline]{%
  \node[anchor=base,inner sep=2pt,outer sep=0pt,fill = #2!20] {#1};
    }%
}%
